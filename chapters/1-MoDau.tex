\section{Lý do chọn đề tài}

Trong bối cảnh đô thị hóa nhanh chóng tại Thành phố Hồ Chí Minh, hệ thống giao thông đang đối mặt với những thách thức nghiêm trọng:

\begin{itemize}
    \item Số lượng phương tiện tăng nhanh (9,4 triệu phương tiện vào tháng 11/2024)
    \item Tình trạng ùn tắc giao thông gia tăng (tăng 17\% tại khu trung tâm)
    \item Ô nhiễm môi trường nghiêm trọng (13 triệu tấn CO$_2$ mỗi năm)
    \item Mức độ ô nhiễm không khí cao gấp 4 lần tiêu chuẩn WHO
\end{itemize}

Các phương pháp điều khiển giao thông truyền thống không còn phù hợp, do đó việc nghiên cứu các giải pháp công nghệ tiên tiến như \ac{drl} là hết sức cần thiết.

\section{Mục tiêu, đối tượng và phạm vi nghiên cứu}
\subsection{Mục tiêu nghiên cứu}


\subsection{Đối tượng nghiên cứu}


\subsection{Phạm vi nghiên cứu}


\section{Phương pháp nghiên cứu}


\section{Cấu trúc Khoá luận tốt nghiệp}
\label{sec:CauTruc}
Khóa luận được trình bày bao gồm 5 chương. Nội dung của tóm tắt từng chương được trình bày như sau:
\begin{itemize}
    \item \textbf{Chương 1: Mở đầu}: Thông tin tổng quan về đề tài mà chúng tôi đã thực hiện
    \item \textbf{Chương 2: Cơ sở lý thuyết}: Các thông tin về định nghĩa, cơ sở lý thuyết liên quan tới các thành phần trong đề tài
    \item \textbf{Chương 3: Phương pháp thực hiện}: Mô tả các mô hình, quy trình và các phương pháp thực hiện
    \item \textbf{Chương 4: Thực nghiệm, đánh gía và thảo luận}:Trình bày các bước thực hiện đề tài qua đó đưa ra đánh dựa trên kết quả đạt được 
    \item \textbf{Chương 5: Kết luận và định hướng phát triển}:Đưa ra kết luận từ các đánh giá và định hướng phát triển cho đề tài
\end{itemize}