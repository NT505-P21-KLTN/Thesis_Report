\section{Lý do chọn đề tài}

Sự gia tăng nhanh chóng của phương tiện giao thông trên đường đã dẫn đến nhiều thách
thức nghiêm trọng như tắc nghẽn giao thông, vấn đề an toàn và tác động tiêu cực
đến môi trường. Điều khiển tín hiệu giao thông đóng vai trò quan trọng trong việc
giải quyết các thách thức này thông qua việc cải thiện an toàn giao thông đường bộ,
giảm thiểu tắc nghẽn và giảm khí thải từ phương tiện. Theo truyền thống, hệ
thống điều khiển đèn giao thông dựa trên phương pháp thời gian cố định sử dụng
phương pháp Webster, dựa vào dữ liệu giao thông lịch sử để xác định thời gian tín
hiệu cố định, không thay đổi theo điều kiện giao thông thực tế. Các hệ thống điều
khiển biến động cung cấp sự linh hoạt hơn bằng cách điều chỉnh thời gian tín hiệu
dựa trên sự hiện diện của phương tiện theo thời gian thực, nhưng vẫn phụ thuộc
vào kế hoạch thời gian được xác định trước. Hệ thống điều khiển tín hiệu giao thông
thích ứng - \ac{atsc} đã xuất hiện như một giải pháp hiệu quả hơn bằng cách điều
chỉnh thời gian tín hiệu theo thời gian thực để thích ứng với nhu cầu giao thông
động. Các hệ thống ATSC truyền thống như \ac{scats} và \ac{scoot} đã chứng minh hiệu
quả, nhưng vẫn bị hạn chế bởi việc phụ thuộc vào thời gian cài đặt trước và các
mô hình giao thông đơn giản hóa. Để khắc phục những hạn chế này, phương pháp \ac{drl}
đã được đề xuất. Khác với các phương pháp truyền thống, \ac{drl} không yêu cầu
kiến thức trước về các kịch bản giao thông cụ thể mà học hỏi chính sách tối ưu
thông qua tương tác liên tục với môi trường giao thông. Các hệ thống \ac{atsc}
dựa trên \ac{drl} được phát triển chủ yếu để giải quyết tình trạng ùn tắc giao thông
và tối ưu thời gian chờ của các phương tiện tham gia giao thông.
\section{Mục tiêu, đối tượng và phạm vi nghiên cứu}
\subsection{Mục tiêu nghiên cứu}
Mục tiêu nghiên cứu của đề tài bao gồm:
\begin{itemize}
    \item Tích hợp thuật toán nhận diện vật thể \ac{yl} vào hệ thống camera để
        thu thập các dữ liệu cần thiết cho mô hình \ac{drl}

    \item Xây dựng mô hình \ac{drl} điều khiển đèn giao thông thông minh, có khả
        năng điều phối hiệu quả với các điều kiện giao thông khác nhau

    \item Xây dựng giao diện để người quản lý có thể chủ động điều phối và theo dõi
        luồng giao thông
\end{itemize}

\subsection{Đối tượng nghiên cứu}
\begin{itemize}
    \item Hệ thống điều khiển đèn giao thông

    \item Thuật toán học tăng cường \ac{drl}

    \item Thuật toán nhận diện vật thể \ac{yl}

    \item Môi trường giả lập giao đông đô thị - \ac{sumo}
\end{itemize}
\subsection{Phạm vi nghiên cứu}

\section{Phương pháp nghiên cứu}
Phương pháp nghiên cứu của nhóm là sử dụng môi trường giả lập giao thông đô thị \ac{sumo}
từ dữ liệu thu thập được từ hệ thống camera tại các giao lộ được tích hợp thuật
toán nhận diện vật thể \ac{yl}. Ngoài ra, nhóm cũng sử dụng môi trường giả lập \ac{sumo}
làm môi trường huấn luyện cho mô hình \ac{drl}. Tiếp theo, nhóm sẽ tiến hành với
2 kịch bản kịch bản. Trong kịch bản nghiên cứu đầu tiên, nhóm nghiên cứu tập trung
vào việc giả lập môi trường giao thông tại một giao lộ và tiến hành so sánh phương
pháp sử dụng mô hình \ac{drl} với phương pháp điều khiển truyền thống. Ngoài ra,
kịch bản này còn được dùng để kiểm chứng tính hiệu quả của quá trình mô phỏng dữ
liệu từ môi trường giao thông thực tế sang môi trường giả lập thông qua thuật
toán nhận diện vật thể \ac{yl}. Trong kịch bản thứ hai, nhóm sẽ triển khai trên 4
giao lộ có liên kết với nhau. Mỗi giao lộ sẽ hoạt động tương tự như trong kịch
bản thứ nhất tuy nhiên sẽ có một server trung tâm thực hiện để trao đổi thông
tin giữa các agent và cung cấp một giao diện quản lý cho người dùng có thể quan
sát tình trạng của các Agent.

\section{Cấu trúc Khoá luận tốt nghiệp}
\label{sec:CauTruc} Khóa luận được trình bày bao gồm 5 chương. Nội dung của tóm tắt
từng chương được trình bày như sau:
\begin{itemize}
    \item \textbf{Chương 1: Mở đầu}: Thông tin tổng quan về đề tài mà chúng tôi đã
        thực hiện

    \item \textbf{Chương 2: Cơ sở lý thuyết}: Các thông tin về định nghĩa, cơ sở
        lý thuyết liên quan tới các thành phần trong đề tài

    \item \textbf{Chương 3: Phương pháp thực hiện}: Mô tả các mô hình, quy trình
        và các phương pháp thực hiện

    \item \textbf{Chương 4: Thực nghiệm, đánh gía và thảo luận}:Trình bày các
        bước thực hiện đề tài qua đó đưa ra đánh dựa trên kết quả đạt được

    \item \textbf{Chương 5: Kết luận và định hướng phát triển}:Đưa ra kết luận
        từ các đánh giá và định hướng phát triển cho đề tài
\end{itemize}