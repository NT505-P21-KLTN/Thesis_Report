Nghiên cứu này đã phát triển thành công một hệ thống điều khiển đèn giao thông thông minh sử dụng Deep Reinforcement Learning, đạt được những kết quả vượt trội so với các phương pháp truyền thống.

\section{Đóng góp chính của nghiên cứu}

\subsection{Hệ thống DQN Single-Agent hiệu quả}
Nghiên cứu đã chứng minh thành công khả năng áp dụng DQN cho điều khiển giao lộ đơn:
\begin{itemize}
    \item \textbf{Hiệu suất cao:} Cải thiện 14.3\% thời gian chờ và cải thiện 10.5\% độ dài hàng đợi so với hệ thống cố định
    \item \textbf{Huấn luyện ổn định:} Đạt được độ ổn định cao trong quá trình huấn luyện với mô hình Balanced
    \item \textbf{Khả năng triển khai:} Tạo ra nền tảng vững chắc cho phát triển hệ thống đa tác nhân
\end{itemize}

\subsection{Hệ thống Sync Agent}
Phát triển thành công hệ thống điều phối đa giao lộ với kiến trúc phân tầng DQN-SAC:
\begin{itemize}
    \item \textbf{Hiệu suất vượt trội:} Cải thiện 27.0\% n thời gian chờ và 23.2\% độ dài hàng đợi, vượt trội 12.6\% so với giao lộ đơn
    \item \textbf{Kiến trúc kết hợp:} Kết hợp DQN (điều khiển cục bộ) với SAC (điều phối toàn cục) trong kiến trúc phân tầng
    \item \textbf{Sẵn sàng triển khai:} 5 mô hình TensorFlow chuyên biệt được tối ưu cho các điều kiện giao thông khác nhau
\end{itemize}

\subsection{Phân tích độ phức tạp giao thông}
Nghiên cứu đầu tiên phân tích định lượng mối quan hệ giữa độ phức tạp giao thông và hiệu suất DQN:
\begin{itemize}
    \item \textbf{Tương quan mạnh:} Xác định mối tương quan nghịch đảo r = -0.94 giữa lưu lượng giao thông và hiệu suất tối ưu hóa
    \item \textbf{Framework triển khai:} Cung cấp hướng dẫn thực tế với kỳ vọng hiệu suất từ 12.4\% (cao điểm) đến 34.5\% (giao thông thấp)
    \item \textbf{Mô hình chuyên biệt:} Phát triển các mô hình chuyên dụng cho từng mức độ phức tạp giao thông
\end{itemize}



\section{Hạn chế và hướng phát triển}

\subsection{Hạn chế hiện tại}
\begin{itemize}
    \item \textbf{Xác thực thực tế:} Cần kiểm thử với dữ liệu giao thông thực tế chất lượng cao
    \item \textbf{Quy mô mở rộng:} Đã thử nghiệm với 4 giao lộ, cần mở rộng lên 8-12 giao lộ
    \item \textbf{Điều kiện đặc biệt:} Chưa kiểm thử với thời tiết khắc nghiệt, tai nạn, sự kiện đặc biệt
\end{itemize}

\subsection{Hướng phát triển tương lai}
\begin{itemize}
    \item \textbf{Triển khai thí điểm:} Áp dụng hệ thống trong môi trường thực tế để xác thực hiệu suất
    \item \textbf{Mở rộng quy mô:} Phát triển cho mạng lưới giao thông lớn hơn (10+ giao lộ)
    \item \textbf{Tích hợp tính năng:} Thêm ưu tiên xe cứu thương, thích ứng thời tiết, điều phối người đi bộ
    \item \textbf{Tối ưu đa mục tiêu:} Cân bằng hiệu suất, an toàn và tác động môi trường
\end{itemize}

\section{Kết luận}

Nghiên cứu đã đạt được những kết quả quan trọng:

\begin{enumerate}
    \item \textbf{Ứng dụng thành công DQN:} Áp dụng DQN cho điều khiển giao lộ đơn với hiệu suất cải thiện 14.3\%

    \item \textbf{Phát triển hệ thống Sync Agent:} Hệ thống điều phối đa giao lộ đạt mức cải thiện 27.0\% , vượt trội các phương pháp truyền thống

    \item \textbf{Phân tích độ phức tạp giao thông:} Định lượng mối quan hệ giao thông-hiệu suất (r=-0.94) với hướng dẫn triển khai thực tế
\end{enumerate}

Hệ thống đã sẵn sàng cho giai đoạn triển khai thí điểm với 5 mô hình chuyên biệt và tài liệu kỹ thuật hoàn chỉnh. Nghiên cứu tạo nền tảng vững chắc cho phát triển hệ thống giao thông thông minh tương lai với hiệu suất vượt trội và khả năng mở rộng thực tế.